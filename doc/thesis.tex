\documentclass[nocoverpage,swedish,g5paper]{thesis}
%
%   optional options to documentclass:
%
%   coverpage   : Create both cover, inside front and text.
%                 Useful for web publishing.
% 
%   nocoverpage : Inner part of thesis only, do not create cover sheet.
%                 Useful for printing.
%   
%   onlycoverpage : Only create cover page. Ignores all text.
%                   Useful for printing.  
%
%   onlytext : Only print the text of the work. No cover and no inside front.
%              Useful for proof-reading copies.
%  
%  g5paper, s5paper, a4paper : Choose paper format, 
%
%  9pt, 10pt, 11pt, 12pt : Choose typeface size.
%
%  draft, final : Draft marks errors with a black box in text.
%
%  openright, openany : openright makes chapters only open at
%                       right hand pages.
%
%  * : Anything else is intepreted as the babel name of a
%      foregin language which is applied to the 'foregincommand'.
%
%
%  Default : s5paper,10pt,final,openright
%
%
%
%  required parameters
%
\title{Monte Carlo study of criticality in finite lattice models}
\author{Simon Sandell}
\date{June 2018}
\shortdate{2018}
\type{Master Thesis}
\department{Department of Theoretical Physics,\\School of Engineering Sciences}
\address{SE-106 91 Stockholm, Sweden}
\city{Stockholm}
\country{Sweden}
\publisher{Printed in Sweden by Universitetsservice US AB, Stockholm January 2010}
\copyrightline{\copyright\ Simon Sandell, June 2018}
\trita{FYS-9999:99}
\isbn{999-99-9999-999-9}
\issn{9999-999X}
\isrn{KTH/FYS/-{}-99:99-{}-SE}
\comment{Scientific thesis for the degree of Master in Science of Engineering (MSc) in the subject area of Theoretical physics.\\ \\ \textbf{Cover illustration:} Non-existent.}
%
%  optional parameters
%
\cplogo{\includegraphics[height=2.5cm]{./kthlogo.eps}}
\innerlogo{\includegraphics[height=2.5cm]{./kthlogo.eps}}
\subtitle{An investigation of the discrepancy of measured and calculated critical exponents of the lambda-transition in $^{4}He$, as well as calculation of the critical exponents of the transition in the 3 dimensional Ising-model, becase why not?}
\division{Condensed Matter Theory}
\centercomment{\centerline{Typeset in \LaTeX}}
%\foregincomment{Akademisk avhandling f\"or avl\"aggande av teknologie licentiatexamen (TeknL) inom
%\"amnesomr{\aa}det teoretisk fysik.}
%\dedication{To Someone}

%\usepackage[swedish]{babel}
\usepackage{verbatim}
\usepackage{bm}
\usepackage{enumerate}
\usepackage[dvips]{graphicx}
\usepackage{amsmath,amsfonts,amssymb}
\usepackage[utf8]{inputenc}
\usepackage{url}
\usepackage[square, comma, sort&compress,numbers]{natbib}

\unitlength=1mm

\def\slc#1{\setbox0=\hbox{$#1$}           % set a box for #1
    \dimen0=\wd0                                 % and get its size
    \setbox1=\hbox{/} \dimen1=\wd1               % get size of /
    \ifdim\dimen0>\dimen1                        % #1 is bigger
       \rlap{\hbox to \dimen0{\hfil/\hfil}}      % so center / in box
       #1                                        % and print #1
    \else                                        % / is bigger
       \rlap{\hbox to \dimen1{\hfil$#1$\hfil}}   % so center #1
       /                                         % and print /
    \fi}

\newcommand{\todo}[1]{(\textbf{TODO:} #1)}
\newcommand{\ud}{\mathrm{d}}
\newcommand{\dd}[2]{\frac{{\rm d}#1}{{\rm d}#2}}
\newcommand{\citeb}[1]{[\citen{#1}]}
\newcommand{\Ref}{[{\bf REF}]}
\newcommand{\Fig}{[{\bf FIG}]}
\newcommand{\chk}{[{\bf CHECK}]}
\newcommand{\im}{\mathrm{i}}
\newcommand{\Mpl}{M_{\rm Pl}}
\newcommand{\Mpr}{\bar{M}_{\rm Pl}}
\newcommand{\Ms}{M_*}
\newcommand{\Msr}{\bar{M}_*}
\newcommand{\ie}{{\it i.e.}}
\newcommand{\eg}{{\it e.g.}}
\newcommand{\hc}{{\rm h.c.}}
\newcommand{\trm}[1]{\textrm{#1}}
\newcommand{\oxx}{\omega_{XX'}}
\newcommand{\hamxy}{\sum\limits_{\langle i,j\rangle}\bm{s_i}\cdot\bm{s_j}}

\begin{document}

%\def\@cite#1{[#1]}

\begin{abstract}
Put abstract here.

And here.

And here..
\\\noindent \strut \\
{\bf Key words}: Monte Carlo, Phase Transition, Finite Size Scaling, 3D Ising model, 3DXY model,Critical Exponents,Universality
\end{abstract}

%\begin{otherlanguage}{swedish}
%\begin{foreginabstract}
%\todo{Skriv ett abstract p{\aa} svenska.}
%\\\noindent \strut \\
%{\bf Nyckelord}: Extradimensionella kvantf{\"a}ltteorier, universella extra dimensioner, ADD-modeller, Kaluza--Klein-m\"ork materia, neutrinomassor, LHC-fenomenologi
%\end{foreginabstract}
%\end{otherlanguage}

\begin{preface}
%        File: preface.tex
%     Created: mån feb 12 01:00  2018 C
% Last Change: mån feb 12 01:00  2018 C
%
\section{Preface}
This is the preface.




\end{preface}

\tableofcontents

% This separates the introduction from the main part of the thesis.
\mainmatter


\chapter{Introduction}
%        File: preface.tex
%     Created: mån feb 12 01:00  2018 C
% Last Change: mån feb 12 01:00  2018 C
%

Phase transitions are interesting phenomenae. They can be explained to a layman using everyday examples such as ice melting, water boiling, etc, but the precise behaviour of physical quantities near and at the transtition temperature is often very hard to both calculate and measure.

The superfluid transition in helium got it's nickname, ``The lambda transition'', due to the peculiar shape of the graph of heat capacity versus temperature, which resemples the greek letter. The transition happens at about 2.2 K, at normal pressure, where the liquid transitions into a superfluid state. It is important as a experimental verification of the Renormalization Group-theory (RG), as it is one of few systems where experimental measurement of these quantities is feasible. 

The lambda phase transition in Helium-4 has been studied extensively both in theory, simulations and experiments. In 2003 the results of measurements performed in orbit was published\cite{Lipa2003} stating a value for the critical exponent alpha to unprecedented precision.

The experiment was performed on the Space Shuttle Columbia, the reason for this is that a gravitational gradient will introduce pressure, causing a transition broadening. In theory, the heat capacity should diverge to infinity at the transition temperature.
The measurements of the heat capacity as well as the superfluid density was performed, to temperatures in the range of $nK$ of $T_c$, with no evidence of broadening.

Theoretical calculations of Sokolov and Nikitina \cite{Sokolov2016} have given predictions of the critical exponents which to the stated precision agrees with the value seen in experiment.
However, numerical studies performed by Campostrini et. al. \cite{Campostrini2006} has predicted values that are incompatible with both experiment and the theoretical predictions. 

The aim of this thesis was to achieve numerical estimations to a higher precision than previous studies. By utilizing a special scaling correction approach, to investigave whether the discrepancy would disappear or emerge more clearly. 

Our results show a better/the same/worse agreement with the experimental value. Agreement shows that numerical studies are a valid method of study of these systems, and discredits the idea that numberical simulations may be part of a different but closely related universality class.




\chapter{Background}\label{ch:Background}
%        File: preface.tex
%     Created: mån feb 12 01:00  2018 C
% Last Change: mån feb 12 01:00  2018 C
%
% TODO: Scaling and universality: more equations.
% scaling laws and definitions of critical exponents, maybe remove? maybe write some text. move to appendix or other place?
% There should be some parts about xy model and ising3d model here.
In this section we will describe the models we simulate using Monte Carlo, as well as establish some theoretical background to explain how physical information can be extracted from simulated systems with sizes $10^X$ times smaller than actual realizations of the transition studied in experiments.
\section{Models} % define equations, define quantities.
The goal of this thesis is to estimate the critical exponents of the lambda phase transition (\lpt) in $^4$He by Monte Carlo (MC) simulation of a model. 
The principle of Universality allows us to freely chose a model for our simulation as long as it is part of the same Universality Class as the \lpt, since all those models are guaranteed to have the exact same critical properties. 
The three criteria that define universality classes are the symmetry group of the order parameter, the physical dimensions and the range of the interactions.
The class of the lamba transition in $^4$He is represented by the symmetry group $O(2)$, short range interactions and $d=3$. With $d=2$ instead the class changes to that of the famous topological Kosterlitz-Thouless transition.

The model we use in our simulation is the so-called XY-model in 3 dimension and allowing only nearest neighbour interactions on a cubic lattice.
We employ periodic boundary conditions, so that the nearest neighbours of any given lattice site $\bm{s} = (x,y,z)$ is defined as 
\begin{align}
    \bm{n_{1,2}} &= (\trm{mod}(L+x\pm1), &   &y,                    &    z)&\\
    \bm{n_{3,4}} &= (x,                  &   &\trm{mod}(L+y\pm1),   &    z)&\\
    \bm{n_{5,6}} &= (x,                  &   &y,                    &    \trm{mod}(L+z\pm1))&
\end{align}
where the lattice indices $(x,y,z)$ range from $[0,L-1]$.

The hamiltonian is written as 
\begin{equation}
  \label{xyham}
  H_{\trm{3DXY}} = -K\sum\limits_{\langle i,j\rangle} \bm{s_i}\cdot\bm{s_j} = -K\sum\limits_{\langle i,j\rangle} \cos(\alpha_i - \alpha_j)
\end{equation}
where $K$ is the coupling constant, which we during the rest of the thesis let $K=1$.
In the XY-model, the spins are confined to a plane, so they can be parametrized by $ \bm{s_1} = (\cos(\alpha_i),\sin(\alpha_i)$, hence the second equality in \ref{xyham}.

Next, we define some of the quantities which we calculate during simulation.
We define the magnetization as 
\begin{equation}
  M = \left| \sum_i \bm{s_i}\right|
  \label{}
\end{equation},
the energy as 
\begin{equation}
  E = -\sum\limits_{\langle i,j \rangle} \cos(\alpha_i - \alpha_j)
  \label{}
\end{equation},
the superfluid stiffness/helicity modulus as 
\begin{equation}
  L\rho_s = -\langle E\rangle - \frac{1}{T}\left\langle\left(\sum_i \sin(\alpha_i - \alpha_{i+(1,0,0)})\right)^2\right\rangle
  \label{}
\end{equation}
The superfluid stiffness/density/helicity modulus is a quantity representing the shift in free energy introduced by a twist of the spins.
To improve statistical results, we also compute
\begin{align}
  L\rho_s = -\langle E\rangle - \frac{1}{T}\left\langle\left(\sum_i \sin(\alpha_i - \alpha_{i+(0,1,0)})\right)^2\right\rangle\\
  L\rho_s = -\langle E\rangle - \frac{1}{T}\left\langle\left(\sum_i \sin(\alpha_i - \alpha_{i+(0,0,1)})\right)^2\right\rangle
  \label{}
\end{align}
In the regular liquid phase, the energy of the system is unaffected by uniform boundary motion. The superfluid stiffness is a quantity which measures the shift in energy due to a shift of the angles along a boundary.
It is interesting as it scales as 

The helicity modulus $\Upsilon(T)$ is a measure of the system to a ``phase twisting'' field, for a superfluid system the helicity modulus is related to the superfluid density
$\rho_s = (m/\hbar)^2 \Upsilon(T)$.
Fischer et al showed in 1973 that this quantity can be calculated within the framework of equilibrium statistical mechanics, but one need to go beyond bulk properties.
Operational definition for spherical model (1952 Barber Fischer) and ideal Bose gas (Barber 1977) under periodic and antiperiodic boundary conditions:
\begin{equation}
  \Upsilon(T) = \trm{lim}\limits_{L\rightarrow \infty} (2L^2/\pi^2) (F^{1/2} (T;L)-F^0(T;L)).
  \label{}
\end{equation}
Rudnick Jasnow (1977) operational definition: 
\begin{equation}
  \Upsilon(T) =\left \frac{  \pa^2F(T;k_0)}{\pa k^2_0}\right|_{k_0 =0}
  \label{}
\end{equation}
Thought: the boundary conditions induce a long wavelength twist of the wavenumber $k_0$ of the order parameter.
Showed that Josephson relation ($\nu = 2\beta -\eta\nu = 2 - \alpha -2\nu = (d -2)\nu$) (1966) is exact using this definition and epsilon-expansion.

\section{Scaling}
Often in physics, relations between different quantities are described by so-called power-laws, expressions on the form $f = f(\xi)\cdot g^{x}$, where the physical quantity $f$ is said to scale with $x$ in relation to some other physical quantity $g$. $\xi$ represents variables that are deemed irrelevant for the scaling behaviour of $f$.

So also in statistical physics, where such laws are fundamental to understanding phase transitions.

In statistical physics, when studying a specific system, the fundamental quantity is some thermodynamic potential,  Gibbs, Helmholtz. From this potential, several other physical quantites are derived, such as the energy, magnetization, and other that may depend on the specifics. 
If the system has a phase transition, often, but not always, since what constitutes a phase transition is not trivial, physical quantites will diverege, and follow certain scaling laws as they do. 
Using mean field theory to calculate these laws gives always a fractional exponent in the laws, but experiments had shown evidence of non-fractional exponents. 
Kadanoff realized that a diverging correlation length implied that there was a relation between the lenght scale at which the order parameter was defined and the coupling constants of an effective Hamiltonian. Altough his block-spin approach does not enable one to compute the critical expoents, it was an important step.
The full theory of Renormalization Group was put forth by Wilson.
The core concept is the renormalization group transformations which takes a Hamiltonian and by some method/rule of coarse-graining clumps together short wave-length degrees of freedom, and defines a new effective hamiltonian describing the long-wavelength degrees of freedom with new coupling parameters for the new lenghtscale.
The name renormalization group is not entirely appropriate, since these transformations are in general complicated and non-linear, thus not always having inverses. But the transformations do have the associative property of groups. Rescaling the system by some length $l_1$ and then rescaling again by some other length $l_2$ should be equivalent to performing the rescaling in the other order.
But so far all we did was remove a finite number of degrees of freedom from out system, how can that explain the sigular behaviour at phase transitions? By repeating the transformations an infinite number of times, singular behaviour can be introduced.
The partition function is what we really want to compute to know everything about a physical system, but that task is most often simply unachievable. The renormalization transformation are also not easy to compute, but the transformation of the coupling constants can be approximated.

\section{Universality}
The theromodynamics of any model; the phase diagram, correlation functions, other quantites etc, may depend on the specific values of coupling parameters in the hamiltonian, symmetries, dimensionality, type of lattice, etc. 
But it turns out that the critical phenomena (phase transitions) only depend on three things, the symmetries of the hamiltonian, the dimensionality and the range of interations ( type of critical point ).

We can study the critical behaviour of say Helium-4, which we know is in the $O(2)$ universality class, by calculation themodynamical averages directly by using Monte-Carlo method on the simulated 3DXY-model, which should have the exact same critical exponents. 
One inconvenience is that one cannot simulate the infinite size 3DXY-model due to computer memory finite-ness, thus one simulates instead finite-size versions of the 3DXY-model. The theory of how finite size systems relate to the infinite size models is called finite size scaling. The normal power-laws aquire correction terms, which needs to be accounted for.


\section{Finite size scaling}
Consider a system with a set of coupling constants $[K]$ and linear finite size $L$. 
The singular part of the free energy scales as
\begin{equation}
  f_s([K],L^{-1}) = l^{-d}f_s([K],lL^{-1}).
  \label{}
\end{equation}
Close to a fixed point of the RG, we can write this equation in terms of right eigenvectors of the linearized RG-transform,
\begin{equation}
  f_s(t,h,K_3,\cdots,L^{-1}) = l^{-d}f_s(tl^{y_t},hl^{y_h},K_3 l^{y_3},\cdots,lL^{-1}).
  \label{}
\end{equation}
It is evident that the inverse size of the system is in fact a relevant eigenvector with eigenvalue $y_L = 1$, and only becomes irrelevant in the thermodynamic limit $L^{-1}\rightarrow 0$.
The models we study in this paper have the external field set to zero, if we let the scaling parameter $l = L$, we can write the scaling form of the free energy as
\begin{equation}
  f_s(t,L^{-1}) = L^{-d} F_f(t L^{1/\nu})
  \label{}
\end{equation}
where we used $\nu = 1/y_t$ and the scaling law $2 -\alpha = d\nu$.
$F_f(tL^{1/\nu})$ is the form function for the free energy.
From this equation we can derive the finite size scaling behaviour of any physical quantities by utilizing their relation to the free energy. 
For example, we can find the scaling form of the specific heat as
\begin{equation}
  c_V = \frac{\partial ^2 f_s}{\partial t^2} \sim L^{-d +2/\nu}F_{c}(tL^{1/\nu}) = L^{\alpha/\nu} F_{c}(tL^{1/\nu}).
  \label{}
\end{equation}
This form function will have a maximum at some shifted value of $T$, 
\begin{equation}
  T_c(L) = T_c +  a_0 L^{-1/\nu}.
  \label{}
\end{equation}
This kind of correction is necessary to go from quantities simulated on finite systems to the infinite lattice universal quantities.


Higher order corrections can be included as well.
Similarly for the susceptibility,
\begin{align}
  \chi = \frac{\partial^2 f}{\partial h^2}
  \sim L^{-d +2y_h}F_{\chi}(tL^{1/\nu}) = L^{2-\eta}F_{\chi}(tL^{1/\nu})
  \label{}
\end{align}
where we used that $ 2(d-y_h) = d -2 +\eta$.
The magnetization goes as
\begin{equation}
  M = \frac{\partial f}{\partial h} \sim L^{-d + y_h} = L^{-\beta/\nu}
  \label{}
\end{equation}
The 4th order Binder cumulandt is in the thermodynamic limit constant at $T_c$, and thus scales as 
\begin{equation}
  B \sim a_B(1+L^{-\omega})
  \label{}
\end{equation}
with our ansatz.
The derivative of the 4th order Binder cumuland scales as
\begin{equation}
  \frac{\der B}{\der T}\sim L
  \label{}
\end{equation}

\section{Scaling laws}
\begin{align}
  y_t &= \frac{1}{\nu}\\
  y_h &= \frac{2-\eta +d}{2}\\
  \Delta &= y_h/y_t\\
	2 - \eta &= \frac{\gamma}{\nu} = d\frac{\delta -1}{\delta+1}\\
	\nu d &= 2-\alpha = 2\beta + \gamma = \beta(\delta+1) = \gamma\frac{\delta +1}{\delta -1}
\end{align}
\section{Definitions of critical exponents}
For $h=0$,
\begin{align}
  f &\sim l^{-d}f(tl^{y_t},hl^{y_h},K_3l^{y_3},\cdots,L^{-1}l)\\
  C &\sim t^{-\alpha}\\
  M &\sim (-t)^{\beta}\\
  \chi &\sim |t|^{-\gamma}\\
  \xi &\sim |t|^{-\nu}\\
\end{align}  
For $t =0$,
\begin{align}
  h &\sim m^{\delta}\\
  \langle m(r)m(0)\rangle &\sim r^{-d +2 -\eta}
  \label{}
\end{align}


\chapter{Method}\label{ch:Method}
%        File: preface.tex
%     Created: mån feb 12 01:00  2018 C
% Last Change: mån feb 12 01:00  2018 C
%
% TODO: expand on monte carlo method. write about necessity of quality RNG's  
\section{Monte Carlo}
In the theoretical framework of statistical physics, physical predictions come from thermal averages, calculated from 
\begin{equation}
  \langle A\rangle = \frac{1}{Z}\trm{Tr} e^{-H/T} = \sum_x A(x) P(x)
  \label{MC:eq:avg}
\end{equation}
where  $P(x) = (1/Z)\trm{exp}(-H(x))$ is the Boltzmann distribution, and the sum is a sum over all possible states of the system. 
Models of physical systems are described by writing down the mathematical expression for the Hamiltonian.
An expression for the partition funciton $Z$ can be written down, but for most models, the number of states are infinite or at least impractically large, so the first formula above can seldom be used in practice to calculate the averages.
Estimation of such very large integrals can be done in several ways.
One way is to calculate the value of the function to be integrated at equidistant points inside the integration area and extrapolating to obtain a measure of the full integral.
This method can be biased, if you do not know your function with full detail, you may select points that are somehow correlated or otherwise introduces an unnecessary error.
A better and safer way to obtain an estimate of the integral is to use randomly distributed points.
This will remove potential bias and will often yeild a better estimate when using the same number of points.

In the field of statistical physics, . By simulating a model, and updating the state of the model randomly, but with such rules so that it's states are Boltzmann-distributed, one can estimate the thermal averages by calculating the corresponding quantity of the simulated model at different 'time-steps'.

When implementing such an algorithm, it is convenient to use a pseudo-random number generator (pRNG) to produce random numbers.
The quality of the pRNG is very important when implementing a Monte Carlo-method.
Since one often wants to generate a large number of random numbers, it is important that the generator produces numbers that are not correlated to the previous ones, as this could lead to bias. 
In this work we use a pRNG implemented in the standard library of C++, the mt\_19937\_64.

%The Boltzmann distribution may be unsuitable for some systems/configurations we want to study. In practice, we can use any distribution we want, if we introduce the proper normalization.
%\begin{equation}
 % \langle A \rangle = \frac{\frac{1}{N}\sum_y \frac{A(y) e^{-H(x)/T}}{P'(y)}}{\frac{1}{N}\sum_y \frac{e^{-H(y)/T}}{P'(y)}}  
%\end{equation}
 %where the states $y$ are distributed according to $P'(y)$.

%Importance sampling is a way of reducing the error in the Monte Carlo estimate. Instead of choosing sample points distributed uniformly, one can try to choose the most probable states. Since thermal systems are Boltzmann-distributed, we want a way to randomly generate states distributed according to the Boltzmann distribution.

%//\section{Markov process}
%//A Markovian process is one for which the probility of getting to any state in the system is determined solely by the current state. I.e. the probability of going from state $x_i$ to $x_{j}$ is given by a transition probability $w(x_i \rightarrow x_j)$. The master equation describing the evolution of the probability distribution can then be written $P(x_{i+1}) = k
%// 
%//

\section{Histrogram extrapolation}
Histogram extrapolation is a method used to extrapolate data obtained from a simualtion.
By calculating additional averages during the simulation proceeds, one can obtain data for a range of temperatures surrounding the temperature the simulation is running at.
The averages are calculated as
\begin{equation}
  \langle A \rangle_{\beta_0} =  \frac{\sum_{x}A_x e^{-\beta_0 H_x}}{\sum_{x}e^{-\beta_0 H_x}}
\end{equation}
To get the average of A at some different temperature, say $\beta_1$, we can do
\begin{equation}
  \langle A \rangle_{\beta_1} =  \frac{\sum_{x}A_x e^{-\beta_1 H_x}}{\sum_{x}e^{- \beta_1 H_x}} = \\
  = \frac{\sum_x A_x e^{-(\beta_1 - \beta_0)H_x} e^{-\beta_0 H_x}}{\sum_x e^{-(\beta_1 - \beta_0)H_x}e^{-\beta_0 H_x}} = \\
  = \frac{\langle A e^{-(\beta_1 - \beta_0)H}\rangle_{\beta_0}}{\langle e^{-(\beta_1 - \beta_0)H}\rangle_{\beta_0}}.
\end{equation}
The range of temperatures to which extrapolation is viable will decrease with system size,
as loss in precision will occur when the exponential factor becomes large. To minimize the exponential factor, one can shift the energy of the system by a constant, which we call $H_{max}$. A constant shift in energy does not affect the values, 
\begin{equation}
  \langle A \rangle = \frac{e^{\beta H_{max}}}{e^{\beta H_{max}}}\frac{\sum_x A_x e^{-\beta H_x}}{\sum_x e^{-\beta H_x}}= \frac{\sum_x A_x e^{-\beta(H_x-H_{max})}}{\sum_x e^{-\beta(H_x-H_{max})}}.
  \label{}
\end{equation}
This technique can extend the range of temperate, but fluctuations in energy grow with the system size, so for high enough system sizes, extrapolation becomes fruitless.
The data gathered at the extrapolated temperature will be worse and worse the further we go, since we generate states that are probable for actual running temperature, but these states will likely become more and more unimportant for the extrapolated temperature.
\section{Wolff Algorithm}
The Wolff algorithm is a non-local update method, and is an improvement upon the Swendsen-Wang method in which spin bonds are scanned and either deleted or frozen. This will divide the lattice into clusters, which are then flipped/updated with certain probabilites. In the Wolff algorithm only a single cluster is generated, with the starting spin uniformly random. That way of initializing the cluster, one can expect to hit large clusters with higher probability, since probability of hitting a cluster should depend on its size.
Thus one can avoid the many single spin clusters usually generated in the Swendsen-Wang method.
For a markovian update method to give a chain of system configurations with distribution according to the Boltzmann-factor, it must satisfy two conditions, 

(i) for any given state of the simulated system, any other state must be reachable in a finite number of steps.

(ii) Non-periodicity, it should not be possible to return to a preivous state immediatley, only after a number of steps, $t =nk,~ n = 1,2,3\dots$.
This hold for the Wolff method, since a valid cluster consists of just shifting a single spin with a random angle, thus any configuration can be reached in $N \leq N_{\trm{spins}}$ steps. Also, since the first spin selected is always flipped, the condition of Non-periodicity is also fulfilled.
The algorithm must also have a transition probability configured so that after a long time, the desired stationary distribution is reached, in our case the Boltzmann-distribution.
Call the desired staionary distribution $\rho(X)$, the transition probability $T(X \rightarrow X')$ and the probability of state $X$ at markov timestep t $\rho(X,t)$.
Then the master equaion is 
\begin{equation}
  \rho(X, t+1) - \rho(X,t) = -\sum_{X'} T(X\rightarrow X')\rho(X,t) +\sum_{X} T(X'\rightarrow X)\rho(X',t).
\end{equation}
Then solving for the stationary solution, we get
\begin{equation}
  \sum_{X'}T(X\rightarrow X') \rho(X) =  \sum_{X}T(X'\rightarrow X) \rho(X')
\end{equation}
which is hard to solve in general, but with a very famous paritcular solution called the detailed balace solution can be found,
\begin{align}
  \frac{T(X\rightarrow X')}{T(X'\rightarrow X)} = \frac{\rho(X')}{\rho(X)} 
  \label{eq:detbal}
\end{align}
for all states $X,~X'$.

This solution can be decomposed as
\begin{equation}
  T(X\rightarrow X') = \omega_{XX'} A_{XX'}
\end{equation}
where $\oxx$ represents a trial probability and $A_{XX'}$ represents an acceptance probability.
We let $\omega_{XX'}$ satisfy
\begin{align}
  \oxx = \omega_{X'X}\\
  0 \leq \oxx \leq 1\\
  \sum_{X'} \oxx = 1.
\end{align}
Inserting this form of $T$ into the detailed balance equation then gives
\begin{equation}
  \frac{A_{XX'}}{A_{X'X}} = \frac{\rho_{X'}}{\rho_{X}}.
\end{equation}
In the metropolis algorithm one chooses the acceptance probability as follows
Step 1: From state $X$ propose a new trial state $X'$ with probability given by $\oxx$. Then accept that state with probability 
\begin{equation}
  A_{XX'} = 
  \begin{cases}
    	1 &\trm{ if } \frac{\rho(X')}{\rho(X)} \geq 1\\
	\frac{\rho(X')}{\rho(X)}&\trm{ if } \frac{\rho(X')}{\rho(X)} < 1 .
  \end{cases}
\end{equation}
The trial probability is usually just a uniform distribution.
\section{Wolff on the 3DXY-model}
In the 3DXY-model, each spin is characterized by a single value, their angle $\alpha$.
We define the hamiltonian of the 3DXY-model as 
\begin{equation}
  H = \beta\sum_{<s_i,s_j>} \cos(\alpha_i - \alpha_j)
\end{equation}
where $<s_i,s_>$ denote that only sites $s_i$ and $s_j$ which are nearest neighours should be included.
The probability of adding spin to the cluster is
\begin{equation}
 P(\alpha_i,\alpha_j,\alpha_u) = 1 - \exp(2\beta \cos(\alpha_j - \alpha_u)\cos(\alpha_i - \alpha_u))
  \label{pflip}
\end{equation}
where $\bm{u}$ is the normal to the plane through which the spins are reflected when added.
When a spin is added to the cluster, it is reflected through a plane.
The Wolff-algorithm is defined as follows for the 3DXY-model:
\begin{enumerate}[(i)]
\item Select a starting spin with uniform probability. Select a random plane with normal $\bm{u}$, with the angle $\alpha_u$  to reflect spins through. Reflect the starting spin through the plane, taking it to 
$$\alpha \rightarrow R(\alpha,\alpha_u) = \pi + 2\alpha_u - \alpha$$

and mark it as part of cluster.

\item Add it's neighbours to a list of perimeter spins, to be tried for inclusion in the cluster.

\item Pick out any element in the list. Try to add it to the cluster with probability as defined above. If it is accepted, add it to the cluster. Add nearest neighbours that are not part of the cluster already to the perimeter list.

\item Repeat (iii) until the perimeter list is empty.
\end{enumerate}
For the Wolff-algorithm, to show that it satisfies detailed balance, condiser a 
state $X$ and a state $X'$ differing by a flip of the cluster $c$. 
The transition probabilities obey
\begin{align}
  \frac{T(X\rightarrow X')}{T(X'\rightarrow X)} &= \prod_{\langle s_i s_j \rangle \in \bm{\partial} c} \frac{ 1- P(R(\alpha_i,\alpha_u),\alpha_j,\alpha_u)}{1- P(R(\alpha'_i,\alpha_u),\alpha'_j,\alpha_u)}
\end{align}
where the product is over nearest neighbour bonds where $s_i \in x, s_j \notin c$.
Thus we have that $ R(\alpha_i,\alpha_u) = \alpha'_i $, $R(\alpha'_i) = \alpha_i$ and $\alpha_j = \alpha'_j$.
We get
\begin{align}
  \frac{T(X\rightarrow X')}{T(X'\rightarrow X)}&= \prod_{\langle s_i s_j \rangle \in \bm{\partial} c} \frac{ \exp\left\{2\beta\cos(R(\alpha_i,\alpha_u) - \alpha_u)\cos(\alpha_j - \alpha_u)\right\}}{ \exp\left\{2\beta\cos(R(\alpha'_i,\alpha_u) - \alpha_u)\cos(\alpha'_j -\alpha_u)\right\}} = \\
  &= \prod_{\langle s_i s_j \rangle \in \bm{\partial} c} \frac{ \exp\left\{2\beta\cos(\alpha'_i - \alpha_u)\cos(\alpha'_j - \alpha_u)\right\}}{ \exp\left\{2\beta\cos(\alpha_i,\alpha_u) - \alpha_u)\cos(\alpha_j -\alpha_u)\right\}} = \\
  &= \frac{\rho(X')}{\rho(X)},
\end{align}
and so, detailed balance is satisfied.

\section{Techniques to correct for scaling}
From the chapter of finite size scaling, we have that

\begin{align}
  b'(L) \equiv B(2L,t=0) - B(L,t=0) &= b_1(2^{-\omega} -1)L^{-\omega}\\
  b''(L)\equiv \frac{b'(2L)}{b'(L)} &= 2^{-\omega}\\
  \omega &= -\frac{\ln(b''(L))}{\ln(2)}.
\end{align}
Note that this only holds at the critical temperature, the $\textit{constants}$ $b_0, b_2$ are functions of temperature. Thus if we plot this function using at least four different system sizes, we can get several graphs that intersect at the critical temperature, and from that the scaling correction $\omega$ can be read of without the need for any form of parameter fitting. This depends on the assumption that the higher order corrections are sufficiently small, which they might not in fact be for the lowest system size we have simulated, $L=4$. This method also fails when $b'(L)$ and $b'(2L)$ differ in sign, since then $log(b''(l)$ will be undefined.
For such cases one can resort to methods of parameter fitting.
Consider the Superfluid density, which scales as 
\begin{align}
  r(L)&\equiv \rho_s(L,t=0) \cdot L = r_0 + r_1L^{-\omega_\rho}\\
  r(2L) - r(L) &= r_1(2^{-\omega_\rho} -1)L^{-\omega_\rho}
\end{align}


For the method using 2 system sizes, one can check by visual inspeciton each different value of omega to see for which omega the graphs intersect mostly at one point. One can also make a numerical check, by finding the intersecitons of all curves, and calculating  and plotting the standard deviation of the intersections for each omega.
The files containing the $L^{\omega}*(2L\rho_x(2L) - L\rho_x(L))$ vs $T$ is separately loaded for each calculated value of $\omega$, then intersections of these 3-5 lines are found. 
The intersection finder finds all intersections, 3 in the case of 3 lines, 6 for 4 lines 10 for 5 lines etc, and then calulates the average position of these intersections. The mean of the euclidean distances are then used as a measure of how close the intersections are. This value is then printed together with the value of omega, and finally plotted vs omega.
Hopefully the graph will have a clear minimum where the intersections are closest, thus determining the scaling correction omega.
For low systemsizes higher order corections can distort the found value, thus we visually inspect the graphs to judge how small system sizes should be included. 
\subsection{Constant subraction}
From the scaling hypothesis, 
\begin{align}
  b_L(t=0) &= \frac{\langle M^4 \rangle}{\langle M^2\rangle^2} = A + B\cdot L^{-\omega}\\
  v_L &\equiv b_L(t=0) -A = B\cdot L^{-\omega}.
\end{align}

This tells us that if we plot this new quantity $v_L$ on a log-log plot as a function of the system size $L$, it should be a constant functions with slope equal to negative omega.
If we instead plot $-\frac{\ln\left(\frac{v_{2L}}{v_L}\right)}{\ln(2)} = \omega$ versus the temperature, and for many combinations of L, we should get an intersection at omega, at Tc.
These methods requires that we know the constant $A$.
We can get a good starting estimation for A by extrapolation,
we plot the value of the binder parameter at the intersecion poins, versus the inverse of the system, and draw the line to $1/L = 0$.

\section{Parallell programming}
Simulating time required grows roughly as $L^3$ with system size, one way to speed up data gathering is to parallellize the simulations.
For the problems at hand, we chose to implement the trivial version of parallellization; run many simulations serially on many cores.
The Wolff algorithm can in fact be parallellized \cite{Kaupuzs2010}, but this did not come to our knowledge before a lot of code had been written, so we chose not to implement it. 

The main part of computations/simulations were performed on resources provided by the Swedish National Infrastructure for Computing (SNIC) at the PDC Center for High Performance Computing.
Some less demanding computations were performed on the Octopus-cluster at the Department of Theoretical Physics at KTH.

The parallellization was implemented using MPI, having a master process handling data output while the children ran simulations.
A process that is relatively slow to perform for a computer is to write data to disk. 
The startup cost of a file write is quite high, so if output is not printed as produced, but rather stored up in memory and then printed in larger chunks, cpu time can be saved. 
In effect, the larger the chucks the more efficent the program runs, but one is limited by the available amount of RAM when deciding chunk size.
Depending on computer system, the program may slow considerably or even crash should the available RAM be depleted during a run.
\section{Data analysis}
The analysis of the data from simulations are handled separately.
The implementations were written in Python, utilizing the NumPy and SciPy libraries. 
The figures where produced using the open source program Grace available at ``plasma-gate.weizmann.ac.il/Grace/''.

\section{Implementation}
The numerical method we used to perform the Monte-Carlo simulations was a Wolff algorithm written in C++. Since both the 3DXY-model and the 3D Ising model suffer from critcal slowing down, i.e. the equilibration time tends to infinity at the critical temperature, 
Both the 3DXY-model and the 3D Ising-model suffer from a phenomenon called critical slowing down. Near the transition temperature, the correlation length diverges, so thermal fluctuations propagate very long distances, and so the system takes a very long time to equilibrate to the equilibrum value.
One can wastly improve the quality-data versus cpu-time ratio by utilizing a global update algorithm such as the Wolff algorithm. 
\section{Error in estimation}
The Monte-Carlo method gives estimations of thermal averages,
\begin{equation}
  \langle A \rangle = \frac{1}{N}\sum_i A_i +\delta A.
  \label{}
\end{equation}
If the estimations $A_i$ are stochastically independent, the error can be estimated by
\begin{equation}
  \delta A \approx \frac{\sigma_A}{N-1}.
  \label{}
\end{equation}
This method works nicely for simple averages, no bias can arise and computation of the standard deviation is straight-forward and not computationally intense.
In our analysis we want to calculate functions of averages, such as the binder cumulant, heatcapacity, etc.
A way to do is to compute
$f(A) \approx \frac{1}{N}\sum_i f(A_i)$, and use the same method as above for error estimation.
But that is a bad way.
Functions of averages can produce bias, take for instance the fuction $f(X) = X^2$. Using an estimate  we get $ f(\bar{X} + \delta X) = (\bar{X})^2 + \bar{X}\delta X + (\delta X)^2$ where the last term introduces a positive bias.
To mitigate these biases one wants to first calculate the estimations of the averages as precisely as possible to minimize the error, as otherwise it can build up.
The above method of calculating the error estimate works as well, but a nicer way to do it is by using the jackknife-method.
One of it's nice features is that it can produce error-bars for functions of averages without having to calculate averages of functions of averages.
The method is a resampling method, one divides the data set into blocks, $x_\{i\}$, and then calculate the jackknife-average, defined as 
\begin{equation}
  x_i^J = \frac{1}{N-1}\sum_{j\neq i}x_j.
\end{equation}
The jackknife estimate of any function of averages is then defined as 
\begin{equation}
  f(X)\approx \bar{f^J}\equiv \frac{1}{N}\sum_{i=1}^{N} f(x_i^J).
  \label{}
\end{equation}
and the uncertainty in this estimate is defined as 
\begin{equation}
  \sigma_{f(\bar{X})} \equiv \sqrt{N-1}\sigma_{f^J}
\end{equation}
where
\begin{equation}
  \sigma^2_{f^J} = \bar{(f^J)^2} - (\bar{f^J})^2.
  \label{}
\end{equation}
The size of blocks can be chosen freely, but the result will depend on it. Chosing a smaller blocksize will increase the runtime of the algorithm.



\chapter{Results}\label{ch:Results}
%        File: preface.tex
%     Created: mån feb 12 01:00  2018 C
% Last Change: mån feb 12 01:00  2018 C
%
\section{Results}
\subsection{Data Analysis}
The simulation program outputs raw data ordered in lines. Data from one ``simulation'' is contained in $N_{temperatures} \cdot N_{averages per simulation} \cdot N_{simulations}$.
To calculate quantities of interest from this raw data, a python script has been written, utilizing the numpy library. All available raw data is loaded into a numpy ndarray, and sorted by system size and temperature. Then for each block of data with one temperature and one system size, combined averages are calculated and from these, further functions of the averages are calculated, such as the Binder Parameter, it's derivative, Susceptibility, Superfluid density, etc. To estimate the statistical error in these estimations, we utilize a resampling method usually refered to as the jackknife method. For each block used in calculating the estimations, a jackknife function is called, which calculates the quantities again, but with a subblock of data omitted.
\subsection{3DXY-model}
Simulations were performed using the Wolff-cluster algorithm at the temperature $T_{run} = 2.2020~K$. Histogram extrapolation was performed to extrapolate to the range of temperatures $T_{range} = 2.2015-2.2030 ~ K$.
The simulations are structured so that a lattice is initialized in the zero temperature state, all spins pointing in one direction. Then clusterupdates are performed until over 100k sweeps have been performed.

(We say that one sweep has been performed when a number of spins equal to the total number of spins on the lattice has been tested to be added to a cluster.
Thus the number of clusters and sweeps performed as warmup varies slightly between independent simulations)

After the warmup, cluster updates are performed for 100k more sweeps, and the thermodynamic quantities are collected after each update. Then, averages are calculated and printed. Then 100 sweeps are performed to take the lattice to a state not correlated, and another 100k sample sweeps are performed. This is repeated 100 times. The system sizes simulated are 4, 8, 16, 32, 64 and 128. 
The number of simulations currently used to calculate $U_4, \rho_s, \chi $ etc. for the system sizes 4


\begin{table}[htpb]
\begin{center}
\begin{tabular}{l l l}
  L & No. simulations & No. uncorrelated averages\\
  4 & 110 & 11000\\
  8 & 110 & 11000\\
  16 & 45.54 & 4554\\
  32 & 12 & 1200\\
  64 & 1.16 & 116 \\
  128 & 0  & 0 \\
\end{tabular}
\end{center}
\caption{Number of simulations performed on the 3DXY-model}
\end{table}

\begin{table}[htpb]
\begin{center}
\begin{tabular}{l l l}
  L & No. simulations & No. uncorrelated averages\\
  4 & 100 & 10000\\
  8 & 100 & 10000\\
  16 & 100 & 10000\\
  32 & 97.52 & 9752 \\
  64 & 5.8 & 580 \\
  128 & 1.15  & $115^{*}$ \\
\end{tabular}
\end{center}
\caption{Number of simulations performed on the Ising3D-model  \textsuperscript{*The simulation of 128 is not doing so well, as evident in the plots}}
\end{table}
\subsection{Ising3D model}
The same general formula is used when simulating the Ising3D model, however, the temperature range is changed to capture the phase transition, $T_{\trm{run}} = 4.50, ~ T_{\trm{range}} = 4.486-4.515$.



%

\begin{figure}[!htpb]
  \centering
  \includegraphics[width=\textwidth]{./plots/3DXY/vsT/Susceptibility.eps}
  \caption{3DXY Susceptibility}
\end{figure}


\begin{figure}[!htpb]
  \centering
  \includegraphics[width=\textwidth]{./plots/3DXY/vsT/Magnetization.eps}
  \caption{3DXY Magnetization}
\end{figure}

\begin{figure}[!htpb]
  \centering
  \includegraphics[width=\textwidth]{./plots/3DXY/vsT/Energy.eps}
  \caption{3DXY Energy}
\end{figure}

\begin{figure}[!htpb]
  \centering
  \includegraphics[width=\textwidth]{./plots/3DXY/vsT/omega.eps}
  \caption{3DXY omega}
\end{figure}

\begin{figure}[!htpb]
  \centering
  \includegraphics[width=\textwidth]{./plots/3DXY/vsT/BinderCumulant.eps}
  \caption{3DXY BinderCumulant}
\end{figure}

\begin{figure}[!htpb]
  \centering
  \includegraphics[width=\textwidth]{./plots/3DXY/vsT/dBdT.eps}
  \caption{3DXY dBdT}
\end{figure}

\begin{figure}[!htpb]
  \centering
  \includegraphics[width=\textwidth]{./plots/3DXY/vsT/SuperfluidDensity.eps}
  \caption{3DXY SuperfluidDensity}
\end{figure}

\begin{figure}[!htpb]
  \centering
  \includegraphics[width=\textwidth]{./plots/3DXY/vsT/HeatCapacity.eps}
  \caption{3DXY HeatCapacity}
\end{figure}

\begin{figure}[!htpb]
  \centering
  \includegraphics[width=\textwidth]{./plots/3DXY/vsT/subtract_twoL_Superfluid.eps}
  \caption{3DXY subtract twoL Superfluid}
\end{figure}

\begin{figure}[!htpb]
  \centering
  \includegraphics[width=\textwidth]{./plots/3DXY/vsT/subtract_twoL_Binder.eps}
  \caption{3DXY subtract twoL Binder}
\end{figure}

\begin{figure}[!htpb]
  \centering
  \includegraphics[width=\textwidth]{./plots/3DXY/vsO/three_L_BinderCumuland.eps}
  \caption{3DXY three L BinderCumuland}
\end{figure}

\begin{figure}[!htpb]
  \centering
  \includegraphics[width=\textwidth]{./plots/3DXY/vsO/nu.eps}
  \caption{3DXY nu}
\end{figure}

\begin{figure}[!htpb]
  \centering
  \includegraphics[width=\textwidth]{./plots/3DXY/vsO/eta.eps}
  \caption{3DXY eta}
\end{figure}

\begin{figure}[!htpb]
  \centering
  \includegraphics[width=\textwidth]{./plots/3DXY/vsO/intersection.eps}
  \caption{3DXY intersection}
\end{figure}

\begin{figure}[!htpb]
  \centering
  \includegraphics[width=\textwidth]{./plots/3DXY/vsO/tc.eps}
  \caption{3DXY tc}
\end{figure}

\begin{figure}[!htpb]
  \centering
  \includegraphics[width=\textwidth]{./plots/3DXY/vsO/three_L_SuperfluidDensity.eps}
  \caption{3DXY three L SuperfluidDensity}
\end{figure}

\begin{figure}[!htpb]
  \centering
  \includegraphics[width=\textwidth]{./plots/3DXY/vsL/bin_subtraction_2L.eps}
  \caption{3DXY bin subtraction 2L}
\end{figure}

\begin{figure}[!htpb]
  \centering
  \includegraphics[width=\textwidth]{./plots/3DXY/vsL/Susceptibility.eps}
  \caption{3DXY Susceptibility}
\end{figure}

\begin{figure}[!htpb]
  \centering
  \includegraphics[width=\textwidth]{./plots/3DXY/vsL/Magnetization.eps}
  \caption{3DXY Magnetization}
\end{figure}

\begin{figure}[!htpb]
  \centering
  \includegraphics[width=\textwidth]{./plots/3DXY/vsL/Energy.eps}
  \caption{3DXY Energy}
\end{figure}

\begin{figure}[!htpb]
  \centering
  \includegraphics[width=\textwidth]{./plots/3DXY/vsL/BinderCumulant.eps}
  \caption{3DXY BinderCumulant}
\end{figure}

\begin{figure}[!htpb]
  \centering
  \includegraphics[width=\textwidth]{./plots/3DXY/vsL/dBdT.eps}
  \caption{3DXY dBdT}
\end{figure}

\begin{figure}[!htpb]
  \centering
  \includegraphics[width=\textwidth]{./plots/3DXY/vsL/rs_subtraction_2L.eps}
  \caption{3DXY rs subtraction 2L}
\end{figure}

\begin{figure}[!htpb]
  \centering
  \includegraphics[width=\textwidth]{./plots/3DXY/vsL/SuperfluidDensity.eps}
  \caption{3DXY SuperfluidDensity}
\end{figure}

\begin{figure}[!htpb]
  \centering
  \includegraphics[width=\textwidth]{./plots/3DXY/vsL/HeatCapacity.eps}
  \caption{3DXY HeatCapacity}
\end{figure}







\chapter{Summary and conclusions}\label{ch:Summary}
%        File: preface.tex
%     Created: mån feb 12 01:00  2018 C
% Last Change: mån feb 12 01:00  2018 C
%
% TODO: write about result, draw inferences ( this raises confidence in monte carlo methods for simulation of critical behaviour, shows that new scaling method is good)
This is the summary.




% This starts the appendices.
%\appendix

%\chapter{Appendix about something}
%\input{appendix}

\addcontentsline{toc}{chapter}{Bibliography}
\bibliographystyle{thesis_bib_style}
\bibliography{references}

%\part{Scientific papers}

%\input{prepapers}

\end{document}





