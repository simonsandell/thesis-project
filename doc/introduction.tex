%        File: preface.tex
%     Created: mån feb 12 01:00  2018 C
% Last Change: mån feb 12 01:00  2018 C
%
% TODO: Write something interesting about ising3d and new scaling method.

Phase transitions are interesting phenomenae. They are part of everyday life, with such manifestations as ice melting, water boiling, etc, but there are endlessly many more examples in nature, for example superfluid and superconducting phaste transitions. The best and most accurate theory describing phase transitions of thermodynamical systems is called Renormalization Group-theory (RG) and has been very successfull in explaining critical phenomenae, but the precise behaviour of physical quantities near and at the transtition temperature where the theory makes it's claims are for many systems hard both to calculate from the theory and to measure in experiments.

The superfluid transition in helium got it's nickname, ``The lambda transition'', due to the shape of the graph of heat capacity versus temperature, which resemples the greek letter. The transition happens at about 2.2 K, at normal pressure, where the liquid transitions into a superfluid state. It is important as a experimental verification of the Renormalization Group-theory (RG), as it is one of few systems where experimental measurement of these quantities is feasible. 

The lambda phase transition in Helium-4 has been studied extensively both in theory, simulations and experiments. In 2003 the results of measurements performed in Earth orbit on the space shuttle Columbia was published\cite{Lipa2003} stating a value for the critical exponent alpha to unprecedented precision. 

Gravitational gradients on Earth will induce a pressure gradient in the sample, which will cause transition broadening, which makes high precision measurements impossible.
The measurements of the heat capacity as well as the superfluid density was performed, to temperatures in the range of $nK$ of $T_c$, with no evidence of broadening.

Theoretical calculations of Sokolov and Nikitina \cite{Sokolov2016} have given predictions of the critical exponents which to the stated precision agrees with the value seen in experiment.
However, numerical studies performed by Campostrini et. al. \cite{Campostrini2006} has predicted values that are incompatible with both experiment and the theoretical predictions. 

The aim of this thesis is to achieve numerical estimations to a higher precision than previous studies. By virtue of a new approach to finite size scaling corrections, which has not been widely utilized before, we hope to achieve estimations for the critical exponents which agree with experimenal and theoretical values. The method is very general and is implementable on numerous systems. We chose to investigate the 3DXY model since it is of  particular interest to see if discrepancies will dissapear with higher precision, and also the 3D Ising-model, since it is also a very important model, and it's implementation is very similar to that of the 3DXY-model.

Our results show a better/the same/worse agreement with the experimental value. Agreement shows that numerical studies are a valid method of study of these systems, and discredits the idea that numberical simulations may be part of a different but closely related universality class.


