%        File: preface.tex
%     Created: mån feb 12 01:00  2018 C
% Last Change: mån feb 12 01:00  2018 C
%

\section{Introduction}
The superfluid transition in helium got it's nickname, ``The lambda transition'', due to the peculiar shape of the graph of heat capacity versus temperature, which resemples the greek letter. 

The transition happens at about 2.2 K, at normal pressure, where the liquid transitions into a superfluid state.

The lambda phase transition in Helium-4 has been studied extensively both in theory, simulations and experiments. In 2003 the results of measurements performed in orbit was published\cite{Lipa2003} stating a value for the critical exponent alpha to unprecedented precision.

The experiment was performed on the Space Shuttle Columbia, the reason for this is that a gravitational gradient will introduce pressure, causing a transition broadening. In theory, the heat capacity should diverge to infinity at the transition temperature.
The measurements of the heat capacity as well as the superfluid density was performed, to temperatures in the range of $nK$ of $T_c$, with no evidence of broadening.

Theoretical calculations of Asdf et al. (pseudo-epsilon expansion techniques\cite{Sokolov2016}) have given predictions of the critical exponents which to the stated precision agrees with the value seen in experiment.

However, the numerical studies performed has predicted values that are incompatible with both experiment and the theoretical predictions. \cite{Campostrini2006} 
In this thesis our aim was to achieve numerical estimation to higher precision than prefious studies, and by utilizing a special scaling correction approach, to investigave whether the discrepancy would disappear or emerge more clearly. 

Our results show a better/the same/worse agreement with the experimental value. Agreement shows that numerical studies are a valid method of study of these systems, and discredits the idea that numberical simulations may be part of a different but closely related universality class.


